\section{Introduction}
In my project, I model a quantum network in which peer nodes are connected to a centralized node, the switch, which is responsible for distributing entanglement amongst the peers.
I implement the simulation in Python using only basic scientific Python libraries and my own custom simulation framework.
The switch generates entanglement between itself and the peers according to the Barret-Kok protocol.
In order to entangle two peers, the switch must first generate entanglement between itself and the peers it wants to entangle, and then perform a Bell state measurement, much like a first generation repeater.
Each peer submits request to become entangled with another peer to the switch, and the switch must determine the following:

\begin{itemize}
    \item The order in which to fulfil requests
    \item When to generate entanglement between itself and a peer
    \item When/if to discard an entanglement between itself and a peer
\end{itemize}

In practice, quantum memories are subject to noise, so the entangled Bell pairs will degrade over time.
I implement different algorithms that manage the above factors, and evaluate how they perform under varying network conditions.
The key performance metrics I will analyze are:
\begin{itemize}
    \item Fidelity of swapped Bell state
    \item Sojurn Time (i.e. latency)
    \item Throughput
    \item Relative Throughput (proportion of fulfilled requests to total number of requests seen in the system)
\end{itemize}
I will evaluate performance of different algorithms on varying network conditions such as:
\begin{itemize}
    \item Link distance
    \item Request rates
    \item Symmetrical vs Asymmetrical networks (varied link distances and/or request rates)
\end{itemize}