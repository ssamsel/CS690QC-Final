\section{Implementation}
Resource management algorithms and entanglement distribution algorithms interoperate, so they are combined into one algorithm implemented as a subclass of \textit{Simulation}.
Each simulation algorithm implements three methods: \textit{matching\_decision}, \textit{entanglement\_decision}, and \textit{idle\_decision},
which are invoked looping in that order until the time exceeds the max interval specified for the simulation.
Since request arrivals are Poisson processes, the simulation operates in an approximation of continuous time, limited by floating point accuracy.
At a high level, \textit{matching\_decision} decides which set of peers to entangle swap,
\textit{entanglement\_decision} manages resources (e.g. entanglement generation, discarding past entanglement, etc.),
and \textit{idle\_decision} handles moving forwards in time.
The behavior of each method is not strict and I chose to implement the simulations this way to allow for reuse of code between different algorithms along with flexibility of algorithm design.
\subsection{Approximating Continuous Time}
Let $T_{sim}$ be the current time at an instant in the simulation.
\subsubsection{Entanglement Generation}
Entanglement generation attempts are Bernoulli processes which means that sampling from the geometric distribution simulates the number of attempts before entanglement is generated.
When the algorithm decides it wants to become entangled with peer $i$, it samples $T_{entgl}(i) = T_{sim}+2\tau_iN$, where $N \sim \text{ Geometric}(\frac{\eta_i^{2}}{2})$.
It stores the sampled $T_{entgl}(i)$, and does not consider itself entangled with the peer until the after $T_{engl}(i)$, which is approximated by $T_{sim} \geq T_{entgl}(i)$.
$T_{engtl}(i)$ cannot be resampled until $T_{sim} \geq T_{entgl}(i)$ and the algorithm is restricted from using any $T_{engtl}(i)$ in any decision it makes, as a real implementation cannot see into the future, nor alter it.
\subsubsection{Request Arrivals}
Requests are considered symmetrical, which means we have request rates $\lambda_{i \leftrightsquigarrow j} = \lambda_{i \rightsquigarrow j} + \lambda_{j \rightsquigarrow i}$ by the additive property of Poisson processes.
Each loop of the algorithm involves advancing forwards in time by some value, let that value be $T_{adv}$.
For each request class $k$, $A_k \sim \text{ Poisson}(\lambda_kT_{adv})$ is sampled.
Then to determine the arrivals times of each of the $A_k$ requests, $A_k$ samples are drawn uniformly in the interval $[0, T_{adv}]$, since arrivals are uniformly distributed in fixed intervals in Poisson processes.
Each of these samples is incremented by $T_{sim}$, sorted, and placed into the queue corresponding to request class $k$.
The algorithm does not ``see'' requests that have arrival times greater than $T_{sim}$.
\subsubsection{Idling}
The \textit{idle\_decison} method determines $T_{adv}$. The algorithm can idle until the next entanglement succeeds, until all entanglements succeed, a fixed amount of time, or until the next request arrives.
If the algorithm's determination of $T_{adv}$ computes to 0, the algorithm will idle until the next request arrives.
If there are no requests in any queue, the simulation samples $T_{adv} \sim \text{ Exponential}\left(\frac{1}{\sum_k\lambda_k}\right)$, since Poisson interevent times exponentially distributed.
It is possible that no requests will arrive in this sampled $T_{adv}$, but since exponential distributions are memoryless, such an event does not affect the distribution.